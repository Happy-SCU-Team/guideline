\documentclass[12pt, a4paper, oneside]{ctexart}

\usepackage{soul}

\title{寝室音量计(草案)}
\author{第八组}
\date{\today}

\begin{document}

\maketitle

\section{引言}
由于寝室里面居住的人比较多,难免有些人会在别人休息的时候发出噪音,
比如在晚上打游戏打high了,就容易忘记控制音量,越叫越大声,影响别人的休息。
所以我们需要一个寝室音量计来监测寝室的音量,
当音量超过一定的值的时候,这个设备就会提醒那个小寝的人注意一下,
不要影响到别人的休息。
\subsection*{此设备的益处}
\begin{itemize}
    \item 直接联系:利于保持寝室的安静,有利于大家的休息。
    \item 人际关系:这样可以避免面子过不去,不好意思提醒别人。
    \item 长远来看:有助于维护寝室的和谐,避免为日后的矛盾埋下伏笔。
\end{itemize}

\section{设计思路}
因为小寝内部相对来说比较好调节,所以我们只考虑小寝与小寝之间的关系。

总体思路是以小寝为单位。如果在休息时间声音过大,
根据麦克风判定是哪个寝室发出的。



\textbf{ 注意:以下内容仍待讨论}

\subsection{安装位置}

\begin{itemize}
    \item 挂在小寝的门外面(这样就需要三个设备)
    \item 只是用一个设备,使用多麦克风阵列,可以判定声音的方向
\end{itemize}

\subsection{提醒方式}
\begin{itemize}
    \item 通过QQ机器人在寝室群@那个寝室的成员里提醒他们说话声音小一些。
    \item 直接向那个寝室发出提示声音
\end{itemize}

\subsection{联网方式}
\begin{itemize}
    \item WiFi连接(SCUNET/eduroam/other)
    \item 蓝牙连接(需要转接设备)
    \item 使用eSIM
\end{itemize}

\subsection{目前设计上的缺陷(疑问)}
\begin{itemize}
    \item 续航能力弱可能导致使用麻烦
\end{itemize}


\section{技术栈}
\subsection{总览}
\begin{itemize}
    \item 物联网端:esp32+麦克风
    \item 服务端:
    \begin{itemize}
        \item 服务器程序编写
        \item 通知:QQ机器人程序编写
    \end{itemize}

\end{itemize}

\subsection{分工合作}

\textbf{单片机端:}
\begin{itemize}
    
    \item 单片机程序开发
    \item 判定声音过大算法设计
    \item 单片机外观(封装)设计
\end{itemize}

\textbf{服务器端:}

\begin{itemize}
    \item 服务器端程序开发
    \item 服务器端QQ机器人开发
    
\end{itemize}

\subsection{代码共享平台} 
Github 



\end{document}